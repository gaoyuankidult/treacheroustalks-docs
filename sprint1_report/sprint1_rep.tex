\documentclass[11pt,a4paper]{article}
\usepackage{hyperref}

\begin{document}
\title{Diplomacy \\
  Sprint 1: ``Get ready to Code!''}
\date{\today}
\author{}
\maketitle

\section{Overview}
\url{http://www.it.uu.se/edu/course/homepage/projektDV/ht11} contains a high level overview of the university course.

\section{Team}
\begin{tabular}{c|rl}
Position     & Name              & E-Mail \\
\hline
Spokesperson & Stephan Brandauer & stephan.brandauer@gmail.com \\
Scrum Master & Jan Daniel Bothma & jbothma@gmail.com
\end{tabular}
\\
\\
We are nine students from {\em Austria, China, England/South Africa, Germany, India, Iran, Sweden, Tajikistan}.

Since the tables in our office are arranged in a triangular shape (which aids communication a lot!),
{\bf ``Bermuda Triangle''} for a name was a logical choice.

\section{Methodology}
We chose to use Scrum in our project and so far are very happy with it.
The daily stand-up meetings are very useful to keep each other updated.
Prioritizing and roughly estimating stories is not too much work and seems to pay off immediately.
The estimated velocity from the first planning meeting was remarkably precise, we where out of tasks
(from the committed stories) one day before the retrospective.
\\
\\
We use Redmine as issue tracker (with backlogs, a Scrum plugin). Since we use backlogs,
we gradually began replacing the physical taskboard. For the next sprint,
the physical taskboard will probably go away.
\\
\\
One problem with scrum, however, is that it takes considerable effort for the scrum master to manage the rest of the team.
This could turn out to be a problem for some student projects since most people want to actively develop.

\section{Tools}
The tool-set we use/decided on includes (so far): 
\\
\\
\label{tbl:tools}
\begin{tabular}{r|l}
Ubuntu 11.04 & operating system \\
Git & version control \\
Gerrit & code reviews \\
Riak & database \\
Nitrogen & webframework \\
Ejabbered & XMPP server \\
gen\_smtp & emailing \\
Buildbot & continuous integration server
\end{tabular}


\section{Sprint 1}

\subsection{``Get ready to code!''}

\subsection{Stories}
\subsubsection{Play a game of Diplomacy as a group}
We played the game and discovered that the rules have many subtleties that are not immediately visible.

\subsubsection{Server Setup}
To be able to code in the next sprint, we need to have several servers up and running,
 which includes setting up a project management system, version control and a code review system.
\\
\\
We decided to use Redmine as our project management system, Git for version control
and Gerrit for code reviewing, based on a mixture of experience, gut feeling,
prototyping and browsing documentation.

\subsubsection{School}
To spread knowledge in the team we decided to have schools for Git, Gerrit and Redmine.
\\
\\
One team member held the school and we did some exercises individually.

\subsubsection{Refine Requirements}
The requirements we receive needed to be clarified, grouped and prioritized.
\\
\\
After discussion with Henry we had a better picture of how things in the system should work
and we were able to prioritize them and group them to see what stories are connected.
We will have to further divide the requirements into smaller pieces each sprint. 

\subsubsection{Survey}
We had already done a brief survey to find out what kind of open source systems we could use, 
we still needed to know a bit more about them and how they work.

After doing some experimenting we found that some systems might be more suitable for this project than others.
\\
\\
This results of the survey are summarized in the tools-table on page \pageref{tbl:tools}.

\subsubsection{Legal Game Name}
The name 'Diplomacy' is already protected by copyright law.

Everyone suggested at least one name and we had a vote, the winner was ``Treacherous Talks'', 
and we're currently waiting for approval from our customer.

\subsection{Preliminary Architecture}
To be able to have an idea of what to do during the next sprint, we needed a preliminary architecture.
In order to ensure the highest possible quality of this central topic, 
as well as to achieve common ownership of this very important step, we adopted a peer review process:

Four people formed the original architecture team and came up with a solution. 
After that, a new group formed and tested the architecture by dry-running user stories, 
one member of the original team joined them for clarifications but tried to remain in the background.

Fault tolerance and scalability where top priorities.

\subsection{Retrospective}

\subsection{Opinions}
We collected a few things in the retrospective:
\begin{itemize}
 \item What was good:
  \begin{itemize}
   \item Time boxing
   \item Team-Fika
   \item Good Teamwork
   \item The team supported everyone
   \item Used scrum well for the first sprint
   \item Got a lot of work done
   \item Architecture, Survey, \dots
  \end{itemize}
 \item What could have been done better:
  \begin{itemize}
   \item Longer sprint
   \item Smaller stories/tasks
   \item Think of what you say before the stand-up
   \item Owner of stories, not just tasks
  \end{itemize}
 \item What should be done differently:
  \begin{itemize}
   \item Done criteria on the stories needed
   \item The demo - too chaotic
   \item Update the taskboard
   \item Organize wiki, it starts to become a mess
   \item Clearer stories/tasks
  \end{itemize}
\end{itemize}

\subsection{Improve}
The next sprint will take three weeks; we decided to focus on improving the following:
\begin{itemize}
 \item A clear ``definition of done'' for the stories - during sprint planning.
 \item Better organized demo - one person (Stephan) will moderate and check that every one can demo.
       Every person should stand up and do the demo. Don't demo while sitting.
 \item Update the taskboard immediately! Otherwise people do not know what you're doing and might start the same.
       It is also harder to track the progress
\end{itemize}


\end{document}
