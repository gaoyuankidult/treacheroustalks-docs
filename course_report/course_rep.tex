\documentclass[11pt,a4paper]{report}
\usepackage[utf8]{inputenc}
\usepackage{hyperref}
\usepackage{graphicx}
\usepackage{datetime}
\usepackage{color}

\newcommand{\hi}[1]{{\color{red}\tiny \em #1\/}\\}
\newcommand{\todo}[1]{\footnote{{\color{red} {\bf TODO:} #1}}}

\begin{document}
\title{Course Report \\
  ``Treacherous Talks''}
\date{\today\ @ \currenttime}
\author{Dilshod Aliev, Jan Daniel Bothma, Stephan Brandauer,\\
 Andre Hilsendeger, Rahim Kadkhodamohammadi, Xinze Lin,\\
Tiina Loukusa, Erik Timan, Sukumar Yethadka}
\maketitle
\tableofcontents

\abstract{
Treacherous Talks is an implementation of a board game (``Diplomacy'') as a web
service.

We developed the service completely in erlang while spending lots of time on
optimizing performance and scalability. The project was mostly self organized
by an international team of 9 students and developed using Scrum.
}
\chapter{Introduction}
Diplomacy is a board game, invented in the 1950s where the goal is to try to
conquer Europe just before WW I. You come close to this goal by talking to the
other players --- by diplomacy --- and making them your allies. And you achieve
it by attack them when they do not expect it.

The game is and was commonly played over distance --- starting with playing by
mail, then email and nowadays over pretty web pages with full-blown map
visualization.

The requirements we were faced with asked for implementation of Diplomacy as a
web service while providing several interfaces to this service. Scalability and
Failure Tolerance were of high priority. \\
Even though a board game is a fun thing to implement, we do think that the most
interesting part of our project is the scalability- and fault tolerance-
engineering.
\todo{Make unique - product rep uses same introduction}

\chapter{Resources}
\section{Team}
Since we decided to arrange our workplaces in a triangular shape
\todo{picture of office}, we gave our team the name ``Team Bermuda Triangle''.
This is only one example for how quickly we had formed a well functioning
team. \\
Even though we differed in nationalities as well as ages, we had no
troubles whatsoever to start working together from the start.\\
\\
The team members were: \\

\newcommand{\flag}[1]{\includegraphics[height=10pt]{flags/#1}}

\begin{tabular}{lcrl}
Dilshod Aliev & Tajikistan & \flag{tajikistan.png} \\ \hline
Jan Daniel Bothma & South Africa & \flag{south_africa.png} \\ \hline
Stephan Brandauer & Austria & \flag{austria.jpg} \\ \hline
Andre Hilsendeger & Germany & \flag{germany.jpg} \\ \hline
Rahim Kadkhodamohammadi & Iran & \flag{iran.png} \\ \hline
Xinze Lin & China & \flag{china.png} \\ \hline
Tiina Loukusa & Sweden & \flag{sweden.png} \\ \hline
Erik Timan & Sweden & \flag{sweden.png} \\ \hline
Sukumar Yethadka & India & \flag{india.png}
\end{tabular}\\
\\
A list of the team members would not be complete without mentioning
Karl Marklund, the TA overlooking our project.
Karl has been a great help with all kinds of problems, from getting more
hardware to giving us tips and feedback. Thank you!

\section{Equipment}
We had nine development PCs (HP made) supplied from the university, as
well as four more PCs as servers (git, redmine, gerrit, buildbot),
as well as a meeting/presentation workstation. \\
Our office room was equipped with several white boards that proved to be
very valuable, especially in times where much design and architecture was done.
\section{Tools}
\subsection*{Issue Tracking}
We used redmine, a web based issue tracker.
Redmine was useful for several reasons:
\begin{itemize}
\item built-in wiki \\
  we used the wiki as basis for high level documentation of the architecture
\item scrum plugin \\
  Backlogs, a scrum plugin was very valuable since it quickly replaced the
  post-its-based scrum-board solution while providing the same overview.
  The scrum masters had to spend no time on keeping issue tracking and scrum
  board consistant any more.
\end{itemize}
\subsection*{Version Control}
As a version control system, we used git, which was a choice we did not discuss
at all since it was assumed by the teachers.

\subsection*{Continuous Integration}
Buildbot, our tenth team member, proved to be valuable by avoiding integration
problems early on.
Our combination of buildbot and gerrit made it impossible to merge anything
before buildbot had confirmed that all tests passed.
\subsection*{Code Reviews}
We used gerrit, a web based code review tool that makes git safer.
When a developer pushes a change to the repository, she can not push to the
master branch directly but only to gerrit. Gerrit then publishes this change in
a pretty web interface for everyone to see {\bf and} tells buildbot to run
the tests with that change. \\
Now the other developers can review that change by either looking at the changed
files in the webinterface's diff viewer or by pulling them to their local
repositories and try them out.
If a developer finds any problems, she can insert comments directly in the 
web interface for the author to see and give a negative review in order to
prevent merging of the change. \\
When the author has reacted to those comments, updated his change and the
reviewer is finally happy, the reviewer can give the change a positive rating.

If buildbot AND at least one reviewer agree that the change can be merged,
gerrit will do so.
This process proved to be incredibly safe, many bugs were found very early and
our build broke only once during the whole project.
Even though gerrit is not easy to set up, we think that future teams would
profit a lot by using gerrit.

While this report is written, it is being code reviewed as well.

\subsection*{Testing}
We used EUnit for testing and did so without really considering
CommonTest, so we do not know whether that choice was optimal or not.
EUnit worked ok for us, one problem we had with EUnit was that EUnit assumes
an automatic timeout of 5 seconds for all tests. If a test takes a long time
to run, this timeout can become really annoying because buildbot will sometimes
reject changes only because there where several builds running on buildbot at
one time and it therefore was under higher load. To make matters worse, EUnit
does not seem to have one central switch to change that default value but only
lets you change the value for single tests.

\chapter{Project Methodology and Organization}

\section{The Customer}
Our customer was Henry Nyström from Erlang Solutions. Henry's job according to
Scrum was to prioritize the tasks at hand and be at the planning meeting.
What Henry did instead was to act more like a teacher which led to confusion on
our side at the beginning of the project. Once we got accustomed to that,
we were able to work very efficiently for and with him.
Henry's experience in developing erlang systems was extremely useful and we hope
that he learned a little bit through this project as well since some of our
tools were not known to him (in practice) before.
\section{Erlang User Conference!}
We were invited to present our project in poster form at the
Erlang User Conference in Stockholm. The attendance was free of charge for us thanks to Erlang Solutions. \\
In Stockholm many people asked about the project, most questions were on a high
level, though. We did not get direct feedback that changed the project after the
EUC but the talks there were quite relevent to our project and therefore a big
help. \\
It was as well a very fortunate place for us to be as some companies that were
there are actively trying to hire young developers with erlang experience.
\section{Scrum}
For managing our project we used Scrum as in~\cite{kniberg}.
We tried hard to start up a well functioning software process and believe that
we succeeded. Our flavour of Scrum looked as follows: \\ \\
\begin{tabular}{cc}
  when & what \\ \hline
  daily & standup meeting\\
  weekly & fika \\
  beginning of sprint & planning \\
  end of sprint & demo \\
  end of sprint & retrospective \\
\end{tabular}

\subsection{Standup Meeting}
The standup meeting was strictly limited in time: to 15 minutes (although we
rarely reached 15 minutes). The standup meeting was held at 9:00 sharp every day
except for some planning days. In the standup meeting, everyone was giving a
short overview of his/her last day and of the plan for this day: ``Yesterday, I
did \ldots, I had problems with \ldots; today, I am going to \ldots''.
The positive effects of this are quite subtle and it might be easy to think that
such a meeting is not really necessary, especially when everyone is in the same
room:\\
\begin{itemize}
\item it encourages people to talk about their problems!
  Problems are not something one should hide. Many easy solutions for seemingly
  big problems came out of this procedure.
\item people maintain an overview of the project state even in the parts of the
  project they have not touched in ages.
\item since everything is announced before-hand,
  (``today, I am going to \ldots''), bad decisions can be caught, the rule to
  always work on the most important item is enforced and so on.
\item the standup meeting is an ideal opportunity for the scrum master to team
  up people who don't really know what they should work on.
\end{itemize}
\subsection{Fika}
The swedish tradition of having Fika was something that has been practiced by
several teams on this course before us and we unanimously voted to have weekly
Fika, too. Every monday, one member of the team would bring cake or pastry and
we would eat together and have coffee or tea. This was a delicious way to get to
know the team better. \\
In order to make people punctual (or: in order to have more Fika!), we decided
that each time someone is late for the standup meeting, that person has to
bring one extra Fika (or ``punishment Fika'', as we called it).
JD was the clear winner!
\subsection{Planning}
Planning was our least favourite task but nonetheless a very important one.
Our typical planning day looked as follows:
\begin{itemize}
\item 9.00--9.45: {\bf estimation of available person-days,
  prioritizing of backlog stories}\\
  Who will be missing? What else is coming up (presentations, EUC, etc.)?
  What do we want to accomplish in this sprint?
  What is most important to the customer? The customer was missing that task
  often, so we had to guess Henry would think.
\item 10.00--10.45: {\bf breaking down stories into tasks}\\
  Splitting the big stories into smaller tasks that ideally do not depend on
  each other --- but that was often easier said than done.
  Also: setting ``definitions of done'' (DoDs).
\item 11.00--14.30: {\bf estimating stories}\\
  (with breaks!)
  Guessing how long each story would take with the additional information from
  task breakdown and DoDs.
\item 14.30--15.00: {\bf committing to sprint}\\
  Taking the prioritized, estimated stories and committing to as many as the
  available person days allowed.
\item 15.00--16.00: FIKA
\end{itemize}

\subsection{Demo}
The demo was held at the end of each sprint. Since in Scrum to try to produce
something that is potentially shippable in each sprint, the demo is there to
ensure that. We invited teaching staff and customer to the demos and showed them
what we did in the last weeks. The demo was always a nice ending of a sprint for
us, because it in quite relaxed atmosphere and we were more proud of our
product than worried.

\subsection{Retrospectives}
The retrospective was the absolutely last thing of a sprint --- in a team
meeting we talked about what we liked about the last sprint, what we disliked,
and what we would like to do differently in the next sprint. \\
It was very important to focus on getting negative feedback at first, people
were generally reluctant to give it because they did not want to seem rude or
be critizised themselves. This is, why we had the rule in the first one or two
retrospectives, that everyone {\em has\/} to come up with at least one of each
category. This worked quite well and in the end, even negative feedback was
coming quite easily, although personal criticism was very rare if occuring at
all. \\
After we collected feedback (we collected it in form of post-its on a
whiteboard), we ``dot-voted'' --- everyone gets three or four votes and can
distribute them in the form of a dot over the negative items on the board.
Giving multiple dots to one item is allowed. We took the three items with the
most points and tried to fix them in the course of the next sprint.

\section{Quality Management}
\subsection*{Coding Conventions}
Our coding conventions were agreed on in sprint 1 and we enforced them through
code reviews. Obviously some things were overlooked but the code quality
doubtlessly profited a lot from that practice.
The coding conventions governed coding style (eg.\ max-80-char-lines, no
trailing white space) as well as erlang specifics (eg.\ how to use {\tt catch}).
\subsection*{Reviews}
Reviews were done mostly by whoever had time. A good time for a dev to do 
reviews was while buildbot was testing her newest change. Sometimes, though,
reviews were requested or even traded: ``I'll review your change, if you'll
review mine''. \\
At first it was annoying to have to wait for some time before a change was
actually merged. After a while, though, we got used to the fact that nothing is
merged immediately and continued to work on the next task while our old one was
awaiting review.

Reviews were generally well meaning but that does not mean that we accepted bad
code quality. People tended to look away when a function was not documented here
and there but there still were plenty of negative reviews.

\subsection*{Testing}
We took testing very seriously and the considerable time we invested payed off.
A counting on december 7$^{th}$, 2011 showed that 47\% of code lines were testing
code. This big suite of tests made it incredibly easy to verify that a change 
didn't break anything. If developers are secured by a solid test suite, it is
also easier to start working on code that is not so well known because one 
always has a regression test suite that immediately highlights errors.

\section{Timeline}
\subsection*{Sprint 1 --- Sprint Goal: ``Getting Ready to Code''}
\subsubsection{Servers}
We set up a git- and a redmine-server, buildbot was not
a task in this sprint. We started off with JD as scrum master and Stephan as
spokesperson. Scrum was done on a whiteboard with post-its.
As we all were new to the process, it took us a little while to figure out how
to act.
\subsubsection{Surveys}
We tried out some tools/libraries and evaluated how they would fit
our project. The end result of that process was (see Appendix B for details): \\
\\
\begin{tabular}{r|l}
Ubuntu 11.04 & operating system \\
Git          & version control  \\
Gerrit       & code reviews     \\
Riak         & database         \\
Nitrogen     & webframework     \\
Ejabberd     & XMPP server      \\
gen\_smtp    & emailing         \\
Buildbot     & continuous integration server
\end{tabular}
\subsubsection{Git-School}
Since almost everyone was new to git, Erik talked to some slides about git and
we all did a couple assignments with git afterwards.
The other team was invited as well.
\subsubsection{Requirements}
We worked on the requirements a lot: we clarified them, prioritized them and had
our customer sign off the outcome of this process. It was crucial, that the
customer was happy with the result since we knew that this result would shape
the rest of our project.
\subsubsection{Preliminary Architecture}
We opted for a reviewed architecture process and would highly recommend it to
future teams as well: \\
A group of people was working on an architecture proposal for a couple of days.
Then one person of that group walked a review~group through the result. The
review~group then took the requirements and tried to run them through the
architecture. This proved to be very effective but of course work-intensive. \\
Given the importance of the architecture, the amount of work was acceptable.

\subsection*{Sprint 2 --- Sprint Goal: ``CRU Users''}
\subsubsection{Continuous Integration}
We set up buildbot. Making our build process compatible with buildbot was a lot
of work and maintaining the build process was a task in all future sprints.

Even though Continuous Integration was a lot of work to get up and running, the
productivity bonus it gave us more than made up for it in the long run.

\subsubsection{Testing Competency}
Some team members had not much or no real experience in writing tests, no one
had experience in writing tests in erlang. Therefore we chose to hold a
{\em testing school\/} much like sprint 1's git school.

\subsubsection{Load Test Riak}
Since our first design drafts showed that riak would be very central to our
performance, we decided that a quick load test of riak was in order just to
ensure that it `works'. The results where not extremely surprising and therefore
we continued to rely on riak.

\subsubsection{Interfaces}
We laid the foundations for our interfaces: http (websockets), xmpp, smtp.
The DoD was to have them answer a `ping' with a `pong' or anything comparable.

\subsubsection{Registering, Login, \ldots}
We gave users the possibility to register on our system, to login and to update
their user information. First connections from the interfaces down to the
database were made!

\subsubsection{Problems!}
After sprint 1 which went very well, we were a bit too careless in sprint 2's
planning, especially the estimation. For this reason, we committed to too many
stories and were not able to complete them by the end of sprint 2. \\
We vowed to not make this mistake again.

\subsection*{Sprint 3 --- Sprint Goal: ``Play a Basic Game''}
\subsubsection{Erlang User Conference '11}
From this sprint on, JD resigned as our scrum master --- but he had warned us
from the beginning of the project. Andre replaced him with the beginning of
sprint~3.

Attending the EUC'11 was on our agenda. For that we had to make a
poster, which had to pass several stages of reviews. This proved to be very
time~consuming.
\subsubsection{Refactor UIs}
Since the input channels (http, xmpp, smtp) had been developed in parallel in
the last sprint, they were full of code duplication. We changed that and as well
extended them greatly.
\subsubsection{Game Application}
Since we wanted to be able to play games, we worked hard on the Game
Application. This ranged from setting up games as user (stored in the DB),
changing their phases according to a timer to implementing the game rules.

\subsection*{Sprint 4 --- Sprint Goal: ``Find the Limits of Real Games''}
\subsubsection{Load Testing}
We applied high load to our system and measured where the limits are (and how
they look like). The load tests showed us severe but easily fixable bugs that
we had in our system; we would not have found them otherwise. The scalability
was quite satisfactory for a first try, the raw performance however was
degrading over time until after $\approx$1h the backends crashed. We fixed this
easily by a riak-reconfiguration but did not figure out where the problem came
from exactly.

\subsubsection{Messaging}
The messaging-part explains the word ``Real'' in the sprint goal since one
cannot play a good diplomacy game without communicating.\\
The messaging system needed to support offline messages (when a user receives a
message while she is offline, the message should be delivered upon login).

\subsection*{Sprint 5 --- TODO}
\subsection*{Sprint 6 --- TODO}

\chapter{Conclusion}
\section*{To Future Students}
\todo{Group Discussion}
\begin{itemize}
\item  Investing in code quality pays off. Do never, ever, say things like:
``commit this now, we can make it pretty later''. Because you won't!
Do it now or never.
\item Think `product': everything you do is in order to make the product better.
You write high quality code because it makes you more productive, not because
it's cast in stone.
\item Make everyone understand what you are talking about and take your time to
do that.
\item Create an environment where it's ok not to know something so that people
ask instead of trying to cover their problems. Help each other out where you
can.
\item Planning is boring, but still: take your time to do it! It is incredibly
motivating to see that a sprint goal was met after three weeks of hard work.
\item Work transparent: let everyone know what you are doing and why. This way,
you get a lot more valuable feedback and no one can tell you later that what you
did was wrong --- because they knew what you did.
\item Apply Scrum completely. Don't feel bad for saying ``no'' to TA/customer if
you are running out of time and you didn't commit to something.
\item Do Fika! Because it's delicious!
\end{itemize}
\section*{To Teachers/TA}
\todo{Group discussion}

\chapter{References}

\appendix
\todo{double header}

\chapter{Team Statements}
\hi{ (required)

Each group member explains his/her own contributions to the project, maybe half a page each. Make sure that all members agree on these descriptions.}
\section{Dilshod Aliev}
\section{Jan Daniel Bothma}
\section{Stephan Brandauer}
I was the spokesperson of the team from the beginning. As spokesperson,
I saw it as my job to shield the team from distractions as much as possible.

In the beginning, I was doing the survey on zotonic which looked great but too
big for us. Then, I also was part of the architecture team for the initial
draft.

My main topic in sprint 2 was riak. I set up a cluster and was working with the
first benchmark of riak, together with mostly Sukumar.
I committed the initial version of our DB-wrapper module as well.

In sprint 3, I was working on one task only: the rule engine, our implementation
of the rules. During that sprint, I got a lot done but also lost a lot of
overview over the project since I was working on one task, alone.

Sprint 4 started with the generation of test data for the load tests. After
this, my main contribution was necromancer, our fail-safefty module as well as
integration of {\tt necromancer} in the system (Tiina and JD did the
implementation of game-restarting).

Sprint 5 had me mostly covered preparing for the presentation of Review~2.
The feedback was quite good, but in my opinion we were forced to spend far too
much time on tasks that do not contribute to the product (I estimate that the
presentation preparation took roughly 3 person weeks).
\section{Andre Hilsendeger}
\section{Rahim Kadkhodamohammadi}
\section{Xinze Lin}
\section{Tiina Loukusa}
\section{Erik Timan}
\section{Sukumar Yethadka}

\chapter{Survey}
After a brief survey in the early stages of the project, we chose a few to look
at in more detail, to be able to make a fair decision on what might be suitable
for the Diplomacy server.
\section{Web Frameworks}
\subsection{Zotonic}
Zotonic is a CMS based on Erlang. It has a powerful control interface, has
quite a number of modules available such as authentication and chat and it
achieves modularity by following OTP-principles. Zotonic has support for
websockets, and when websockets are not supported by the browser, it will use
Comet instead. Zotonic has been insprired by Nitrogen and it can only be run
on a MochiWeb server.
\subsection{Erlang Web}
Erlang Web is quite simplistic, it has support for multiple web servers and
distribution. It does not have any extras like a chat nor does it have support
for websockets. Erlang Web can be run on INETS and YAWS web servers.
\subsection{Nitrogen}
Nitrogen uses an event-driven model and offers hot code swapping, stability
and scalability. It includes a simple, yet powerful, template system and is
well documented which is important for us as developers. Nitrogen can run on
several web servers; MochiWeb, YAWS and INETS.
\subsection{Chicago Boss}
Chicago Boss is a new web framework, which has not yet reached a stable state.
In it current state, it still offers some nice features, as it supports multiple
databases, hot code swapping and e-mail among other things. It is well
documented but it does not support websockets nor distributed systems.
\subsection{Results and discussion}
When we tried to decide on what web framework to use, we wanted simplicity, as
we had already decided on not to focus on the look and feel and we didn't want
to spend a lot of time on learning a specific web framework.
The project required the use of websockets and if the framework didn't
provide that up front, we would need to implement it ourselves.\\

Chicago Boss was too immature to be an option, as we want something stable that
we won't need to patch up too much and Erlang Web was very basic and didn't
provide many features that we could use, compared to the other ones.

The only web framework that had support for websockets was Zotonic, but it was
quite complex to learn to use it, and it didn't provide the kind of simplicity
we were looking for. We came to the conclusion that it would be of more use to
us to use a web framework that was easy to use and implement websocket support,
rather than learning how to create new modules for Zotonic.\\

We initially decided to use Nitrogen, because we found that it was one of the
more mature and well used web frameworks out of those we looked at. It could
work with both of the web servers we were choosing among, it was easy to use and
it provided some nice and useful features. It didn't have websocket support but
we thought we would be able to implement that.

Unfortunately, we came to the conclusion that implementing websocket support
into Nitrogen is a project on its own, and we decided to not use a web
framework at all. We ended up using only Javascript, which was sufficient for
our use.

\section{Web Servers}
\subsection{MochiWeb}
\subsection{YAWS}
\subsection{Results and discussion}

\section{SMTP Servers}
\subsection{Erlmail}
\subsection{gen\_smtp}
\subsection{erlang-smtp}
\subsection{Results and discussion}

\section{XMPP}
\subsection{ejabberd server}
\subsection{EXMPP client}
\subsection{Results and discussion}

\section{Databases}
\subsection{Mnesia}
\subsection{Riak}
\subsection{CouchDB}
\subsection{Results and discussion}

\bibliographystyle{plain}
\bibliography{course_rep}

\end{document}
