\documentclass[11pt,a4paper]{article}
\usepackage{hyperref}

\begin{document}
\title{Treacherous Talks 
  Sprint 2: ``CRU Users''}
\date{\today}
\author{}
\maketitle

\begin{tabular}{c|rl}
Position     & Name              & E-Mail \\
\hline
Spokesperson & Stephan Brandauer & stephan.brandauer@gmail.com \\
Scrum Master & Andre Hilsendeger & andre.hilsendeger@gmail.com
\end{tabular}
\\
\\
With the end of sprint \#2, there has been a change in the team:
Jan Daniel Bothma is no longer our scrum master, Andre Hilsendeger will
take over from him. We did this in the hope that the benefits and disadvantages
of that position are spread out more evenly across the team.

\section{The Sprint}
\subsection{Plan ..}
The sprint 2 goal was ``CRU Users'' which, for us, entailed {\it C}reating, {\it R}eading, {\it U}pdating through all three interfaces, Web, Mail, IM.

Even though software design does not directly relate to this goal, ``First Draft of a High Level Design'' was a high-priority story.

Our estimated velocity was 89 story points (1 point being a person day), so we 
also fitted two stories regarding game creation and search into the plan for 
this sprint.

\subsection{.. and Reality}

We quickly ran into trouble integrating our tools with Buildbot, our continous 
integration system (since that requires a build process that is able to work 
from scratch). It was especially hard to build and release ejabberd but that 
works now.
We foresaw that this would take lots of time but, unfortunately, not the scale.
Additionally, we use rebar to handle compilation and releasing of our code. 
Rebar has, though, some issues that took a while to figure and iron out.

Additionally, it turned out, that websockets and nitrogen with yaws don't play
together, so we abandoned nitrogen in favour of simple yaws. This came with 
certain cost, as well.

Due to the amount of time we had to spend on the build system, we began lagging 
behind.

What we were able to deliver in the end was, in fact, the actual sprint goal 
but not the game stories. This amounted to 70 story points. As a result of this,
we are going to reduce the focus factor for this sprint but we hope that 
problems on the scale of ejabberd will not arise in the coming sprint.

\subsection{The Stories}
\begin{tabular}{l|c}
First draft of a high level design & done \\
Set up continuous integration & done \\
All team members must have basic testing competency & done \\
Load test Riak & done \\
A user should be able to connect using a web client & done \\
A user should be able to connect using an im client & done \\
The system should communicate with the mail interface using smtp & done \\
A user should be able to register on the system & done \\
A user should be able to log into the server & done \\
A user should be able to update his/her personal information & done \\
A user should be able to setup a game as an organiser & not done \\
A user should be able to search for games based on properties & not done \\
Unix School & not done
\end{tabular}

\subsection{Retrospective}
We voted on the importance of issues (everyone had 4 votes). \\
The results:
\subsubsection*{What should be done differently:}
\begin{itemize}
\item {\bf Time estimation \& planning the sprint (10 votes)}
\item {\bf More people on code review (10 votes)}
\item {\bf Communicate solutions better (7 votes)}
\item Sticking to the process (4 votes)
\item Build process (3 votes)
\item Coding efficiency (2 votes)
\end{itemize}
The bold items are the ones we will focus on in sprint 3.

The issues from last retrospective (clearer ``Definitions of Done'', 
better organized demo, keeping the taskboard updated) were all accepted as 
reached, we therefore have time to focus on the new improvements.

\end{document}
